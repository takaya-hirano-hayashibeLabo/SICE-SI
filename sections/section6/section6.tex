\section{結言}

本研究では, 入力の時間変動に対してSNNの時定数と膜抵抗を動的に変化させるモデルを提案した.
結果として, 提案手法は入力の時間変動が生じた場合でも, SNNの内部状態のダイナミクスの変化を抑制できる特性を示した.
また, 実際の時系列データの分類タスクに適用することで, 基準となるタイムスケールのみの学習で, 入力の時間変動に対して頑健なモデルの構築が可能であった.
今後の展望としては, 分類タスク以外への提案手法の適用が挙げられる.
時系列回帰や生成に本手法を用いることで, より汎用的な使用可能性の提示が考えられる.
% \makeatletter % @が使える
\subsection{SUBTITLE}
ああああああああああああああああああああああああああああああああああああああああああああああああああ
% \makeatletter % @が使える
\subsection{SUBTITLE}
ああああああああああああああああああああああああああああああああああああああああああああああああああ
% \makeatletter % @が使える
\subsection{SUBTITLE}
ああああああああああああああああああああああああああああああああああああああああああああああああああ\cite{tmpRef}