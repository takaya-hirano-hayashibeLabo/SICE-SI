\section{考察}

\begin{figure*}
    \centering
    \parbox{1.0\linewidth}{
        \centering

        \begin{minipage}{.48 \linewidth} % Adjusted width to fit side by side
            \centering
            \includesvg[width=1 \linewidth, inkscapelatex=false]{static/sec5_fr_snn}
            \subcaption{SNNの出力層におけるスパイク密度}
            \label{sec5:fig:snn}
        \end{minipage}
        \hspace{0.02\linewidth} % これがないと横並びにならない
        \begin{minipage}{.48 \linewidth} % Adjusted width to fit side by side
            \centering
            \includesvg[width=1 \linewidth, inkscapelatex=false]{static/sec5_fr_dyna}
            \subcaption{提案手法モデルの出力層におけるスパイク密度}
            \label{sec5:fig:dyna}    
        \end{minipage}    

        \caption{出力スパイク密度の比較}
        \subcaption*{上段 : タイムスケール$a=1.0$, 下段 : タイムスケール$a=10.0$}
        \label{sec5:fig:disc}
    }
\end{figure*}

実験2より提案手法を用いることで入力の時間変動が生じた場合でも, モデルの精度低下を抑えることが可能であった.
このような結果になった原因として, 時定数と膜抵抗を動的に変化させることで, 入力の時間変動によるモデルの出力の変化を抑えられたからだと考えられる.

実験2において, あるテストデータを入力した際の通常のSNNと提案手法の出力の比較を\Fig{sec5:fig:disc}に示す.
横軸は正規化された時間, 縦軸は単位時間あたりの出力スパイク密度である.
また, グラフのそれぞれの番号は出力層のニューロン番号に対応する.
スパイク密度はその値の変化が少ないほど, 入力のタイムスケール変動によるモデル出力への影響が小さいことを表す.
\Fig{sec5:fig:disc}より, 提案手法は通常のSNNと比較して, 出力スパイク密度の変化が小さいことがわかる.
特に, \#4, \#10の活動の大きいニューロンにおけるスパイク密度変化が小さい.
これは, 入力の時間変動に応じて, \refeqn{sec2:eq:condition}の条件をモデルに対して動的に与えることで, 出力の変動を抑制したからだと考えられる.
一方で, 提案手法における\#11, \#6のニューロンは, タイムスケールの変化によって, その活性・不活性が変化している.
このような変化が生じた原因としては, 理想的な入力と実データのタイムスケーリングでは, そのデータの形式が異なるからだと考えられる.
実験1で示した提案手法におけるSNNの内部状態ダイナミクスの特性は, 時刻$t$におけるスパイクが$a$倍のタイムスケーリングによって, 時刻$at$に移動するという理想的な状況で現れたものである.
実験2では実データを用いたため, 理想的なタイムスケールとは異なっていたと考えられる.
結果として, \refeqn{sec2:eq:condition}の条件を与えたとしても, 出力の変動を抑制できないニューロンが生じたと考えられる.
そのため, 理想的な入力と実際の入力の差異を小さくするように入力データ形式を変換するエンコーダを作成することで, より入力の時間変動に頑健なモデルを構築できると考えられる.

% \makeatletter % @が使える
\subsection{実験1 : SNNの内部状態ダイナミクスのタイムスケーリング評価}
ランダムな値で重みを初期化したSNNに対し, 基準となるスパイク列$\bm{o(t)}$と$a$倍のタイムスケーリングしたスパイク列$\bm{o(at)}$を入力する.
その後, $\bm{o(t)},~\bm{o(at)}$が入力されたときの最終層におけるそれぞれの内部状態を比較する.
このとき, \refeqn{sec2:eq:ideal_laplace}の条件をSNNに与えることで, 提案手法$\bm{v_{proposed}(t)}$が理想的な内部状態$\bm{v(at)}$に近づくことを示す (\textbf{\Fig{sec2:fig:inspike}}).
SNNは入出力を1次元とし, ノード数8の全結合層を2層持つモデルを用いた.
ランダムに生成した100通りのスパイク$\bm{o(t)}$を入力し, $\bm{v(at)}$に対する$\bm{v_{LIF}(t)}, ~ \bm{v_{proposed}(t)}$それぞれの平均二乗誤差 (MSE, Mean Squared Error) を計測した.
SNNのそれぞれのパラメータを\Table{sec3:tab:exp1snn}に示す.


% tabularxによって\linewidthで幅指定可能に (required: tabularx)
% \centering\arraybackslashによって, 各セルの中央に値を表示 (required: array)
\begin{table}[htb]
    \centering
    \caption{SNNモデルパラメータ}
    \label{sec3:tab:exp1snn}
    \begin{tabularx}{0.8\linewidth}{>{\centering\arraybackslash}X>{\centering\arraybackslash}X>{\centering\arraybackslash}X}
        \multicolumn{3}{c}{\textbf{SNN architecture}}\\
        \hline
        \textbf{input size}& \textbf{hidden size} & \textbf{output size}\\
        \hline
        1   & 8, 8 & 1 
    \end{tabularx}

    \begin{tabularx}{0.8\linewidth}{>{\centering\arraybackslash}X>{\centering\arraybackslash}X>{\centering\arraybackslash}X>{\centering\arraybackslash}X>{\centering\arraybackslash}X}
        \multicolumn{5}{c}{\textbf{LIF parameters}}\\
        \hline
        $\bm{dt}$&$\bm{v_{rest}}$&$\bm{v_{th}}$&$\bm{\tau}$&$\bm{r}$\\
        \hline
        0.001&0.0&0.01&0.05&1.0
    \end{tabularx}
\end{table}
% \makeatletter % @が使える
\subsection{実験2 時間変動に対するモデル性能評価}

\subsubsection{データセット}
モデルの学習データセットとしてDVSGesture\cite{dvsgesture}を用いた.
このデータセットは11種類のジェスチャーを時系列に記録したものである.
また, データ形式はイベント形式で保存されており, 変化が生じた時刻とピクセル位置がペアとして保存されている.
イベントが生じた時刻・ピクセル位置を1, それ以外を0とすることで容易にスパイク列へ変換することが可能であるため, SNNを用いたタスク評価に適したデータセットである\cite{9207109}.


\subsubsection{時系列認識モデルとその学習}
提案モデルに対して, DVSGesuteデータセットを用いた時系列認識タスクを学習させた.
また, 比較対象としてLSTM, SNN, Parametric-SNN\cite{ParametricSNN}で構成したモデルも同様のタスクを学習させた.
Parametric-SNNとは, 時定数を学習可能としたSNNであり, 推論時は学習によって得られた時定数を用いる.
LSTMで構成したモデルは, ResNet\cite{ResNet}を用いてフレームごとの特徴抽出を行い, LSTMによって時系列認識を行った\cite{CNNLSTM}.
SNNを用いたモデルはSNNで残差接続を実現したMS-ResNet\cite{MSResNet}を用いて構成した.
また, 畳み込み層間のドロップアウト率は0.3とした.
それぞれのモデル構成を\Table{tab:sec3:modelarchi}に示す.
LSTM, SNN, Parametric-SNNは, $0.5,~1.0,~2.0$倍の3種類のタイムスケールを用いて学習を行った.
提案モデルは$1.0$倍速のデータのみで学習を行い, 線形回帰モデルは3種類のデータを用いて回帰を行った.
DVSGestureデータセットは, 3 msでサンプリングされ, $1.0$倍速で約1.0秒の長さのものを用いた.
\begin{table}[htb]
    \centering
    \caption{評価モデル構成}
    \label{tab:sec3:modelarchi}

    % \begin{tabularx}{1.1\linewidth}{>{\centering\arraybackslash}X>{\centering\arraybackslash}X>{\centering\arraybackslash}X>{\centering\arraybackslash}X>{\centering\arraybackslash}X}
    \begin{tabular}{ccccc}
        \multicolumn{5}{c}{\textbf{Model architecture using LSTM}}\\
        \hline
        \textbf{Layer} & \textbf{Type} & \textbf{Input} & \textbf{output} & \textbf{Layer Nums} \\
        \hline
        1 & ResNet & 2x32x32 & 12x16x16 & 3 \\
        2 & ResNet & 12x16x16 & 32x8x8 & 3 \\
        3 & Linear & 2048 & 512 & - \\
        4 & LSTM & 512 & 512 & - \\
        5 & LSTM & 512 & 256 & - \\
        6 & Linear & 256 & 11 & - \\
    \end{tabular}
    
    % \begin{tabularx}{1.1\linewidth}{>{\centering\arraybackslash}X>{\centering\arraybackslash}X>{\centering\arraybackslash}X>{\centering\arraybackslash}X>{\centering\arraybackslash}X}
    \begin{tabular}{ccccc}
        \multicolumn{5}{c}{\textbf{Model architecture using SNN}}\\
        \hline
        \textbf{Layer} & \textbf{Type} & \textbf{Input} & \textbf{output} & \textbf{Layer Nums} \\
        \hline
        1 & MS-ResNet & 2x32x32 & 12x16x16 & 3 \\
        2 & MS-ResNet & 12x16x16 & 32x8x8 & 3 \\
        3 & Linear & 2048 & 512 & - \\
        4 & Linear & 512 & 11 & - \\
    \end{tabular}

    \begin{tabularx}{0.8\linewidth}{>{\centering\arraybackslash}X>{\centering\arraybackslash}X>{\centering\arraybackslash}X>{\centering\arraybackslash}X>{\centering\arraybackslash}X}
        \multicolumn{5}{c}{\textbf{LIF parameters}}\\
        \hline
        $\bm{dt}$&$\bm{v_{rest}}$&$\bm{v_{th}}$&$\bm{\tau}$&$\bm{r}$\\
        \hline
        0.003&0.0&0.1&0.006&1.0
    \end{tabularx}

\end{table}


\subsubsection{評価方法}
提案手法の入力の時間的変動に対する頑健性を評価するために以下の2つの実験を行った.
\begin{itemize}
    \item シーケンス全体を$a$倍した際のモデル精度評価
    \item シーケンスの前半, 後半を$a_1,~a_2$倍した際のモデル性能評価
\end{itemize}

まず, 入力シーケンス全体を$a$にタイムスケーリングし, 入力の時間変動に対する各モデルの性能を評価した.
倍率$a$は, $1.0$から$20.0$を$1.0$刻みに変化させた値とした.
次に, 入力シーケンスの前半を$a_1$倍, 後半を$a_2$倍にスケールすることで, シーケンスの途中で時間変動が生じた場合の各モデルの性能を評価した.
$a_1,~a_2$はそれぞれ\Table{sec3:tab:exp2:2}に示す値を用いた.

\begin{table}[htb]
    \centering
    \caption{倍率$a_1,~a_2$の組み合わせ}

    \begin{tabularx}{0.8\linewidth}{>{\centering\arraybackslash}X>{\centering\arraybackslash}X>{\centering\arraybackslash}X}
        \hline
        \textbf{Case}&$\bm{a_1}$&$\bm{a_2}$\\
        \hline
        A&1.0&5.0\\
        B&5.0&1.0\\
        C&10.0&5.0\\
        D&5.0&10.0
    \end{tabularx}

    \label{sec3:tab:exp2:2}

\end{table}
% \makeatletter % @が使える
\subsection{SUBTITLE}
ああああああああああああああああああああああああああああああああああああああああああああああああああ\cite{tmpRef}