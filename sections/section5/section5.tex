\section{考察}

% \begin{figure*}[t]
    \centering
    \parbox{1.0\linewidth}{
        \centering

        \begin{minipage}{.48 \linewidth} % Adjusted width to fit side by side
            \centering
            \includesvg[width=1 \linewidth, inkscapelatex=false]{static/sec5_fr_snn}
            \subcaption{通常のSNNの出力層におけるスパイク密度}
            \label{sec5:fig:snn}
        \end{minipage}
        \hspace{0.02\linewidth} % これがないと横並びにならない
        \begin{minipage}{.48 \linewidth} % Adjusted width to fit side by side
            \centering
            \includesvg[width=1 \linewidth, inkscapelatex=false]{static/sec5_fr_dyna}
            \subcaption{提案手法モデルの出力層におけるスパイク密度}
            \label{sec5:fig:dyna}    
        \end{minipage}    

        \caption{出力スパイク密度の比較}
        \subcaption*{上段 : タイムスケール$a=1.0$, 下段 : タイムスケール$a=10.0$}
        \label{sec5:fig:disc}
    }
\end{figure*}
 %位置的にsecton4へ移動

実験2では提案手法を用いることで入力の時間変動に対するモデルの精度低下が抑制される結果となった.
この原因は, 時定数と膜抵抗の動的な変化によって, 入力の時間変動によるモデルの出力の変化が抑えられたからだと考えられる.
実験2における, あるテストデータを入力した際の通常のSNNと提案手法の出力の比較を\Fig{sec5:fig:disc}に示す.
横軸は正規化された時間, 縦軸は単位時間あたりの出力スパイク密度である.
スパイク密度はその値の変化が大きいほど, そのニューロンが活動的であることを示す.
また, グラフのそれぞれの番号は出力層のニューロン番号に対応する.
\Fig{sec5:fig:disc}より, 提案手法は通常のSNNと比較して, タイムスケールの変化による出力スパイク密度の変化が小さいことがわかる.
特に, \#4, \#10の活動の大きいニューロンにおけるスパイク密度変化が小さい.
これは, 入力の時間変動に応じて, \refeqn{sec2:eq:condition}の条件をモデルに対して動的に与えることで, 出力の変動が抑制されたからだと考えられる.
一方で, 提案手法における\#6, \#11のニューロンは, タイムスケールの変化によって, その活性・不活性が変化している.
このような変化が生じた原因としては, 理想的な入力と実データのタイムスケーリングでは, そのデータの形式が異なるからだと考えられる.
実験1で示した提案手法におけるSNNの内部状態ダイナミクスの特性は, 時刻$t$におけるスパイクが$a$倍のタイムスケーリングによって, 時刻$at$に移動するという理想的な状況で示した.
一方, 実験2では実データを用いたため, 入力スパイクは理想的なタイムスケーリングとは異なっていたと考えられる.
結果として, \refeqn{sec2:eq:condition}の条件を与えたとしても, 出力の変動を抑制できないニューロンが生じたと考えられる.
そのため, 理想的な入力と実際の入力の差異を小さくするように入力データ形式を変換するエンコーダを導入することで, より入力の時間変動に頑健なモデルの構築が可能だと考えられる.

% \makeatletter % @が使える
\subsection{SUBTITLE}
ああああああああああああああああああああああああああああああああああああああああああああああああああ
% \makeatletter % @が使える
\subsection{SUBTITLE}
ああああああああああああああああああああああああああああああああああああああああああああああああああ
% \makeatletter % @が使える
\subsection{SUBTITLE}
ああああああああああああああああああああああああああああああああああああああああああああああああああ\cite{tmpRef}