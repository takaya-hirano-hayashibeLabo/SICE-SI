\section{考察}

\begin{figure*}
    \centering
    \parbox{1.0\linewidth}{
        \centering

        \begin{minipage}{.48 \linewidth} % Adjusted width to fit side by side
            \centering
            \includesvg[width=1 \linewidth, inkscapelatex=false]{static/sec5_fr_snn}
            \subcaption{SNNの出力層におけるスパイク密度}
            \label{sec5:fig:snn}
        \end{minipage}
        \hspace{0.02\linewidth} % これがないと横並びにならない
        \begin{minipage}{.48 \linewidth} % Adjusted width to fit side by side
            \centering
            \includesvg[width=1 \linewidth, inkscapelatex=false]{static/sec5_fr_dyna}
            \subcaption{提案手法モデルの出力層におけるスパイク密度}
            \label{sec5:fig:dyna}    
        \end{minipage}    

        \caption{出力スパイク密度の比較}
        \subcaption*{上段 : タイムスケール$a=1.0$, 下段 : タイムスケール$a=10.0$}
        \label{sec5:fig:disc}
    }
\end{figure*}

実験2より提案手法を用いることで入力の時間変動が生じた場合でも, モデルの精度低下を抑えることが可能であった.
このような結果になった原因として, 時定数と膜抵抗を動的に変化させることで, 入力の時間変動によるモデルの出力の変化を抑えられたからだと考えられる.

実験2において, あるテストデータを入力した際の通常のSNNと提案手法の出力の比較を\Fig{sec5:fig:disc}に示す.
横軸は正規化された時間, 縦軸は単位時間あたりの出力スパイク密度である.
また, グラフのそれぞれの番号は出力層のニューロン番号に対応する.
スパイク密度はその値の変化が少ないほど, 入力のタイムスケール変動によるモデル出力への影響が小さいことを表す.
\Fig{sec5:fig:disc}より, 提案手法は通常のSNNと比較して, 出力スパイク密度の変化が小さいことがわかる.
特に, \#4, \#10の活動の大きいニューロンにおけるスパイク密度変化が小さい.
これは, 入力の時間変動に応じて, \refeqn{sec2:eq:condition}の条件をモデルに対して動的に与えることで, 出力の変動を抑制したからだと考えられる.
一方で, 提案手法における\#11, \#6のニューロンは, タイムスケールの変化によって, その活性・不活性が変化している.
このような変化が生じた原因としては, 理想的な入力と実データのタイムスケーリングでは, そのデータの形式が異なるからだと考えられる.
実験1で示した提案手法におけるSNNの内部状態ダイナミクスの特性は, 時刻$t$におけるスパイクが$a$倍のタイムスケーリングによって, 時刻$at$に移動するという理想的な状況で現れたものである.
実験2では実データを用いたため, 理想的なタイムスケールとは異なっていたと考えられる.
結果として, \refeqn{sec2:eq:condition}の条件を与えたとしても, 出力の変動を抑制できないニューロンが生じたと考えられる.
そのため, 理想的な入力と実際の入力の差異を小さくするように入力データ形式を変換するエンコーダを作成することで, より入力の時間変動に頑健なモデルを構築できると考えられる.

% \makeatletter % @が使える
\subsection{SUBTITLE}
ああああああああああああああああああああああああああああああああああああああああああああああああああ
% \makeatletter % @が使える
\subsection{SUBTITLE}
ああああああああああああああああああああああああああああああああああああああああああああああああああ
% \makeatletter % @が使える
\subsection{SUBTITLE}
ああああああああああああああああああああああああああああああああああああああああああああああああああ\cite{tmpRef}