\makeatletter % @が使える
\subsection{実験1 SNNの内部状態ダイナミクスのタイムスケーリング評価}
ランダムな値で重みを初期化したSNNに対し, 基準となるスパイク列$\bm{o_{base}}$と基準から$a$倍のタイムスケーリングしたスパイク列$\bm{o_{scaled}}$を入力する.
$\bm{o_{base}},~\bm{o_{scaled}}$が入力されたときの最終層の内部状態$\bm{v_{base}}, ~ \bm{v_{scaled}}$を比較する.
このとき, \refeqn{sec2:eq:ideal_laplace}の条件をSNNに与えることで$\bm{v_{scaled}}$が$\bm{v_{base}}$に近づくことを実験的に示す.
SNNは入出力を1次元とし, ノード数8の全結合層を2層持つモデルを用いた.
ランダムに生成した100通りのスパイク$o(t)$を入力し, $v(at)$に対する$v_{LIF}(t), ~ v_{proposed}(t)$それぞれの平均二乗誤差 (MSE, Mean Squared Error) を計測した.
SNNのそれぞれのパラメータを\Table{sec3:tab:exp1snn}に示す.


% tabularxによって\linewidthで幅指定可能に (required: tabularx)
% \centering\arraybackslashによって, 各セルの中央に値を表示 (required: array)
\begin{table}[htb]
    \centering
    \caption{SNNモデルパラメータ}
    \label{sec3:tab:exp1snn}
    \begin{tabularx}{0.8\linewidth}{>{\centering\arraybackslash}X>{\centering\arraybackslash}X>{\centering\arraybackslash}X}
        \multicolumn{3}{c}{\textbf{SNN architecture}}\\
        \hline
        \textbf{input size}& \textbf{hidden size} & \textbf{output size}\\
        \hline
        1   & 8, 8 & 1 
    \end{tabularx}

    \begin{tabularx}{0.8\linewidth}{>{\centering\arraybackslash}X>{\centering\arraybackslash}X>{\centering\arraybackslash}X>{\centering\arraybackslash}X>{\centering\arraybackslash}X}
        \multicolumn{5}{c}{\textbf{LIF parameters}}\\
        \hline
        $\bm{dt}$&$\bm{v_{rest}}$&$\bm{v_{th}}$&$\bm{\tau}$&$\bm{r}$\\
        \hline
        0.001&0.0&0.01&0.05&1.0
    \end{tabularx}
\end{table}