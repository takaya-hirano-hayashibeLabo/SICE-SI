\makeatletter % @が使える
\subsection{実験2 時間変動に対するモデル性能評価}

\subsubsection{データセット}
モデルの学習データセットとしてDVSGesture\cite{dvsgesture}を用いた.
このデータセットは11種類のジェスチャーを時系列に記録したものである.
また, データ形式はイベント形式で保存されており, 変化が生じた時刻とピクセル位置がペアとして保存されている.
イベントが生じた時刻・ピクセル位置を1, それ以外を0とすることで容易にスパイク列へ変換することが可能であるため, SNNを用いたタスク評価に適したデータセットである\cite{9207109}.


\subsubsection{時系列認識モデルとその学習}
提案モデルに対して, DVSGesuteデータセットを用いた時系列認識タスクを学習させた.
また, 比較対象としてLSTM, SNN, Parametric-SNN\cite{ParametricSNN}で構成したモデルも同様のタスクを学習させた.
Parametric-SNNとは, 時定数を学習可能としたSNNであり, 推論時は学習によって得られた時定数を用いる.
LSTMで構成したモデルは, ResNet\cite{ResNet}を用いてフレームごとの特徴抽出を行い, LSTMによって時系列認識を行った\cite{CNNLSTM}.
SNNを用いたモデルはSNNで残差接続を実現したMS-ResNet\cite{MSResNet}を用いて構成した.
また, 畳み込み層間のドロップアウト率は0.3とした.
それぞれのモデル構成を\Table{tab:sec3:modelarchi}に示す.
LSTM, SNN, Parametric-SNNは, $0.5,~1.0,~2.0$倍の3種類のタイムスケールを用いて学習を行った.
提案モデルは$1.0$倍速のデータのみで学習を行い, 線形回帰モデルは3種類のデータを用いて回帰を行った.
DVSGestureデータセットは, 3 msでサンプリングされ, $1.0$倍速で約1.0秒の長さのものを用いた.
\begin{table}[htb]
    \centering
    \caption{評価モデル構成}
    \label{tab:sec3:modelarchi}

    % \begin{tabularx}{1.1\linewidth}{>{\centering\arraybackslash}X>{\centering\arraybackslash}X>{\centering\arraybackslash}X>{\centering\arraybackslash}X>{\centering\arraybackslash}X}
    \begin{tabular}{ccccc}
        \multicolumn{5}{c}{\textbf{Model architecture using LSTM}}\\
        \hline
        \textbf{Layer} & \textbf{Type} & \textbf{Input} & \textbf{output} & \textbf{Layer Nums} \\
        \hline
        1 & ResNet & 2x32x32 & 12x16x16 & 3 \\
        2 & ResNet & 12x16x16 & 32x8x8 & 3 \\
        3 & Linear & 2048 & 512 & - \\
        4 & LSTM & 512 & 512 & - \\
        5 & LSTM & 512 & 256 & - \\
        6 & Linear & 256 & 11 & - \\
    \end{tabular}
    
    % \begin{tabularx}{1.1\linewidth}{>{\centering\arraybackslash}X>{\centering\arraybackslash}X>{\centering\arraybackslash}X>{\centering\arraybackslash}X>{\centering\arraybackslash}X}
    \begin{tabular}{ccccc}
        \multicolumn{5}{c}{\textbf{Model architecture using SNN}}\\
        \hline
        \textbf{Layer} & \textbf{Type} & \textbf{Input} & \textbf{output} & \textbf{Layer Nums} \\
        \hline
        1 & MS-ResNet & 2x32x32 & 12x16x16 & 3 \\
        2 & MS-ResNet & 12x16x16 & 32x8x8 & 3 \\
        3 & Linear & 2048 & 512 & - \\
        4 & Linear & 512 & 11 & - \\
    \end{tabular}

    \begin{tabularx}{0.8\linewidth}{>{\centering\arraybackslash}X>{\centering\arraybackslash}X>{\centering\arraybackslash}X>{\centering\arraybackslash}X>{\centering\arraybackslash}X}
        \multicolumn{5}{c}{\textbf{LIF parameters}}\\
        \hline
        $\bm{dt}$&$\bm{v_{rest}}$&$\bm{v_{th}}$&$\bm{\tau}$&$\bm{r}$\\
        \hline
        0.003&0.0&0.1&0.006&1.0
    \end{tabularx}

\end{table}


\subsubsection{評価方法}
提案手法の入力の時間的変動に対する頑健性を評価するために以下の2つの実験を行った.
\begin{itemize}
    \item シーケンス全体を$a$倍した際のモデル精度評価
    \item シーケンスの前半, 後半を$a_1,~a_2$倍した際のモデル性能評価
\end{itemize}

まず, 入力シーケンス全体を$a$にタイムスケーリングし, 入力の時間変動に対する各モデルの性能を評価した.
倍率$a$は, $1.0$から$20.0$を$1.0$刻みに変化させた値とした.
次に, 入力シーケンスの前半を$a_1$倍, 後半を$a_2$倍にスケールすることで, シーケンスの途中で時間変動が生じた場合の各モデルの性能を評価した.
$a_1,~a_2$はそれぞれ\Table{sec3:tab:exp2:2}に示す値を用いた.

\begin{table}[htb]
    \centering
    \caption{倍率$a_1,~a_2$の組み合わせ}

    \begin{tabularx}{0.8\linewidth}{>{\centering\arraybackslash}X>{\centering\arraybackslash}X>{\centering\arraybackslash}X}
        \hline
        \textbf{Case}&$\bm{a_1}$&$\bm{a_2}$\\
        \hline
        A&1.0&5.0\\
        B&5.0&1.0\\
        C&10.0&5.0\\
        D&5.0&10.0
    \end{tabularx}

    \label{sec3:tab:exp2:2}

\end{table}