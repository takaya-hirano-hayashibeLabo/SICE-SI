\makeatletter % @が使える
\subsection{実験2 時間変動に対するロバスト性評価}

\subsubsection{データセット}
モデルの学習データセットとしてDVSGesture\cite{dvsgesture}を用いる.
DVSGestureデータセットは11種類のジェスチャーを時系列に記録したデータセットである.
また, データ形式はイベント形式で保存されており, ジェスチャーによって変化が生じた時刻とピクセル位置がペアとして保存されている.
イベントが生じた時刻・ピクセル位置を1, それ以外を0とすることで容易にスパイク列へ変換することが可能であるため, SNNを用いたタスク評価に適したデータセットである\cite{9207109}.

\begin{figure*}[t]
    \centering
    \parbox{.9\linewidth}{
        \centering
        \includesvg[width=1 \linewidth, inkscapelatex=false]{static/sec3_gesture}
        \caption{ジェスチャーデータの例 (右腕を大きく回す動作) : イベント形式のため動作の大きい腕の情報が多く記録されている}
        \label{sec3:fig:gesture}
    }

\end{figure*}

\subsubsection{時系列認識モデルとその学習}
提案手法のモデルに対して, DVSGesuteデータセットを用いた時系列認識タスクを学習させる.
また, 比較対象としてLSTM, SNN, Parametric-SNN\cite{ParametricSNN}で構成したモデルについても同様のタスクを学習させる.
Parametric-SNNとは, 時定数を学習可能としたSNNであり, 推論時は学習によって得られた時定数を用いる.
LSTMで構成したモデルは, ResNet\cite{ResNet}を用いてフレームごとの特徴抽出を行い, LSTMによって時系列認識を行った\cite{CNNLSTM}.
SNNを用いたモデルはSNNで残差接続を実現したMS-ResNet\cite{MSResNet}を用いて構成した.
また, 畳み込み層間のドロップアウト率は0.3とした.
それぞれのモデル構成を\Table{tab:sec3:modelarchi}に示す.

\begin{table}[htb]
    \centering
    \caption{評価モデル構成}
    \label{tab:sec3:modelarchi}

    % \begin{tabularx}{1.1\linewidth}{>{\centering\arraybackslash}X>{\centering\arraybackslash}X>{\centering\arraybackslash}X>{\centering\arraybackslash}X>{\centering\arraybackslash}X}
    \begin{tabular}{ccccc}
        \multicolumn{5}{c}{\textbf{Model architecture using LSTM}}\\
        \hline
        \textbf{Layer} & \textbf{Type} & \textbf{Input} & \textbf{output} & \textbf{Layer Nums} \\
        \hline
        1 & ResNet & 2x32x32 & 12x16x16 & 3 \\
        2 & ResNet & 12x16x16 & 32x8x8 & 3 \\
        3 & Linear & 2048 & 512 & - \\
        4 & LSTM & 512 & 512 & - \\
        5 & LSTM & 512 & 256 & - \\
        6 & Linear & 256 & 11 & - \\
    \end{tabular}
    
    % \begin{tabularx}{1.1\linewidth}{>{\centering\arraybackslash}X>{\centering\arraybackslash}X>{\centering\arraybackslash}X>{\centering\arraybackslash}X>{\centering\arraybackslash}X}
    \begin{tabular}{ccccc}
        \multicolumn{5}{c}{\textbf{Model architecture using SNN}}\\
        \hline
        \textbf{Layer} & \textbf{Type} & \textbf{Input} & \textbf{output} & \textbf{Layer Nums} \\
        \hline
        1 & MS-ResNet & 2x32x32 & 12x16x16 & 3 \\
        2 & MS-ResNet & 12x16x16 & 32x8x8 & 3 \\
        3 & Linear & 2048 & 512 & - \\
        4 & Linear & 512 & 11 & - \\
    \end{tabular}

    \begin{tabularx}{0.8\linewidth}{>{\centering\arraybackslash}X>{\centering\arraybackslash}X>{\centering\arraybackslash}X>{\centering\arraybackslash}X>{\centering\arraybackslash}X}
        \multicolumn{5}{c}{\textbf{LIF parameters}}\\
        \hline
        $\bm{dt}$&$\bm{v_{rest}}$&$\bm{v_{th}}$&$\bm{\tau}$&$\bm{r}$\\
        \hline
        0.003&0.0&0.1&0.006&1.0
    \end{tabularx}

\end{table}


LSTM, SNN, Parametric-SNNは, $0.5,~1.0,~2.0$倍の3種類のタイムスケールのものを用いて, 11種類のジェスチャーを分類するモデルを学習させる.
一方, 提案手法のモデルのみ$1.0$倍速のデータだけを用いてモデルの学習を行う.
DVSGestureデータセットは, 3 msでサンプリングされ, $1.0$倍速で約1.0秒の長さのものを用いた.


\subsubsection{評価方法}
提案手法の入力の時間的変動に対する頑健性を評価するために以下の2つの実験を行った.
\begin{itemize}
    \item 入力シーケンス全体を$a$倍した際のモデル精度評価
    \item 入力シーケンスの前半を$a_1$, 後半を$a_2$倍した際のモデル性能評価
\end{itemize}

まず, 入力シーケンス全体を$a$にタイムスケーリングしすることで, 入力の時間変動に対する各モデルの性能を評価した.
倍率$a$は, $1.0$から$20.0$を$1.0$刻みに変化させた値とした.
また, 入力シーケンスの前半を$a_1$倍, 後半を$a_2$倍にスケールすることで, シーケンスの途中で時間変動が生じた場合の各モデルの性能を評価した.
$a_1,~a_2$はそれぞれ\Table{sec3:tab:exp2:2}に示す値を用いた.

\begin{table}[htb]
    \centering
    \caption{倍率$a_1,~a_2$の組み合わせ}

    \begin{tabularx}{0.8\linewidth}{>{\centering\arraybackslash}X>{\centering\arraybackslash}X>{\centering\arraybackslash}X}
        \hline
        \textbf{Pattern}&$\bm{a_1}$&$\bm{a_2}$\\
        \hline
        A&1.0&5.0\\
        B&5.0&1.0\\
        C&10.0&5.0\\
        D&5.0&10.0
    \end{tabularx}

    \label{sec3:tab:exp2:2}

\end{table}