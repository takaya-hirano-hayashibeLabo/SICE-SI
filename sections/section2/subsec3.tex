\makeatletter % @が使える
\subsection{逐次的な時定数と膜抵抗の推定およびSNNへの適用}
入力スパイク列$\bm{o(t)}$に対して\refeqn{sec2:eq:condition}の条件を満たす時定数$\tau',~r'$を逐次的に推定する.
SNNの時定数と膜抵抗を推定した$\tau',~r'$に置き換えることで, 入力のタイムスケール変化に対して頑健な推論を行う.

まず, 入力スパイク列$o(t)$のタイムスケール$a$をスパイク密度$\rho$を用いて推定する.
スパイク密度$\rho$は, ある一定時間$T$のうちに$o(t)=1$であった割合を表す.
図XXXに示すようにタイムスケールの伸縮に対してスパイク密度は変動する.
そのため, スパイク密度$\rho$を用いてタイムスケール$a$の推定が可能だと考えられる.
本研究では線形回帰を用いて近似を行う (\refeqn{sec2:eq:reg}).
\begin{equation}
    \begin{split}
        \rho&=\frac{1}{T} \frac{1}{\Pi_{k=1}^{N}d_k} \sum_t^T \sum_{i_i,..,i_N}o_{t,i_1,..,i_N}\\
        \log{\hat{a}}&=\alpha \log{\rho} + \beta
    \end{split}
    \label{sec2:eq:reg}
\end{equation}
ここで, スパイク列$\bm{o}$は$T \times N$次元のテンソルとし, $d_i$は各次元のサイズである.
また, $\alpha, \beta$は線形回帰によって求めた値であり, $\hat{a}$は推定したタイムスケールである.

次に, 推定した$\hat{a}$を用いて, SNNの時定数と膜抵抗を\refeqn{sec2:eq:condition}を満たす値に変動させる.
基準となるSNNの時定数と膜抵抗を$\tau_{base},~r_{base}$とすると, \refeqn{sec2:eq:replace}のように表される.
\begin{equation}
    \begin{split}
        \tau_{base} \rightarrow \hat{a}~\tau_{base}\\
        r_{base} \rightarrow \hat{a}~r_{base}\\
    \end{split}
    \label{sec2:eq:replace}
\end{equation}