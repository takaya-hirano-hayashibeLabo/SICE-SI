% 図をきれいに配置したいだけならparboxを使う (Fgi1, Fig2としたいとき)
% 関連する図をきれいに配置するときはminipageを使う (Fig1(a), (b)としたいとき)

\begin{figure*}[t]
    \centering

    \parbox{.9 \linewidth}{
        \centering
        % svgのままコンパイルするならこれ. コンパイルは`lualatex --shell-escape`を使う(required: svg, fontspec)
        \includesvg[width=1 \linewidth, inkscapelatex=false]{static/sec2_model}
        \caption{モデル構造}
        \label{sec2:fig:model}
    }
    
    \vspace{1em} % 図の間にスペースを追加

    \parbox{.9\linewidth}{
        \centering
        \includesvg[width=1 \linewidth, inkscapelatex=false]{static/sec2_snn_horizontal}
        \caption{SNNの流れ}
        \label{sec2:fig:snn}
    }
\end{figure*}

\begin{figure*}[t]
    \centering
    \parbox{.9\linewidth}{
        \centering
        \includesvg[width=1 \linewidth, inkscapelatex=false]{static/sec2_laplace}
        \caption{入力スパイクのタイムスケーリング}
        \label{sec2:fig:inspike}
    }

\end{figure*}