\section{手法}

本手法で用いるモデル構造を\Fig{sec2:fig:model}に示す.
本モデルは大きく2つの構成に分かれる.
\begin{itemize}
    \item SNNの時定数・膜抵抗を推定する線形回帰モデル
    \item SNNを用いた推論モデル
\end{itemize}
SNNの内部状態ダイナミクスを決定する時定数や膜抵抗は, 推論時には固定した値を用いることが多い\cite{zheng2024temporal}\cite{ParametricSNN}.
本研究では, これらのパラメータを線形回帰モデルによって推定し動的に変動させる.
これにより, SNNを用いた推論モデルは, 基準となるタイムスケールのみの学習で, 多様なタイムスケールに対する推論が可能となる.
% 図をきれいに配置したいだけならparboxを使う (Fgi1, Fig2としたいとき)
% 関連する図をきれいに配置するときはminipageを使う (Fig1(a), (b)としたいとき)

\begin{figure*}[t]
    \centering

    \parbox{.9 \linewidth}{
        \centering
        % svgのままコンパイルするならこれ. コンパイルは`lualatex --shell-escape`を使う(required: svg, fontspec)
        \includesvg[width=1 \linewidth, inkscapelatex=false]{static/sec2_model}
        \caption{モデル構造}
        \label{sec2:fig:model}
    }
    
    \vspace{1em} % 図の間にスペースを追加

    \parbox{.9\linewidth}{
        \centering
        \includesvg[width=1 \linewidth, inkscapelatex=false]{static/sec2_snn_horizontal}
        \caption{SNNの流れ}
        \label{sec2:fig:snn}
    }
\end{figure*}

\begin{figure*}[t]
    \centering
    \parbox{.9\linewidth}{
        \centering
        \includesvg[width=1 \linewidth, inkscapelatex=false]{static/sec2_laplace}
        \caption{入力スパイクのタイムスケーリング}
        \label{sec2:fig:inspike}
    }

\end{figure*}
\makeatletter % @が使える
\subsection{Spiking Neural Network}
Spiking Neural Network (SNN) は脳の神経細胞のダイナミクスを模倣し, 0か1のスパイク値を入出力とするニューラルネットワークである.
時刻$t$における第$l-1$層から$l$層のSNNへの入力$\bm{o}^{l-1}\left(t\right)$は, \Fig{sec2:fig:snn}に示すように処理され, 次の第$l+1$層へ出力される.

% \begin{enumerate}
%     \item 入力スパイク$\bm{o^{l-1}\left(t\right)}$への重み付け
%     \item 入力に基づくSNNの内部状態の更新
%     \item 内部状態に基づくスパイク$\bm{o^l\left(t\right)}$の出力
% \end{enumerate}


まず, SNNへ入力されたスパイク$\bm{o}^{l-1}\left(t\right)$は\refeqn{eq:input_spike}によって重み付けされシナプス電流$\bm{i}^l\left(t\right)$へ変換される.

\begin{equation}
    \bm{i}^l\left(t\right) = \bm{W}^l\bm{o}^{l-1}\left(t\right) + \bm{b}^l
    \label{eq:input_spike}
\end{equation}
ここで, $\bm{W}^l, \bm{b}^l$はそれぞれ第$l$層のニューラルネットワークの重みとバイアスである.

次に, 第$l$層のSNNの内部状態$\bm{v}^l\left(t\right)$は, 神経細胞の活動をモデル化したLeaky Integrate-and-Fire (LIF) モデル (\refeqn{eq:lif}) によって更新される.

\begin{equation}
    {\tau}\frac{d\bm{v}^l\left(t\right)}{dt}=-\left(\bm{v}^l\left({t-1}\right)-v_{rest}\right)+r\bm{i}^l\left(t\right)
    \label{eq:lif}
\end{equation}
ここで, $\tau$は神経細胞の時定数, $v_{rest}$は内部状態の初期状態, $r$は神経細胞の膜抵抗である.

最後に, 内部状態$\bm{v}^l\left(t\right)$が一定の閾値$v_{th}$を超えたときに出力スパイク$\bm{o}^l\left(t\right)$が1となって出力される (\refeqn{eq:outputSpike}).
また, 閾値を超えた内部状態は初期状態へとリセットされる (\refeqn{eq:outputSpike2}).
\begin{equation}
    \begin{split}
      \bm{o}\left(t\right)^{l}&=\left\{
        \begin{alignedat}{2}
          1 &\:\left(\bm{v}^l\left(t\right){\geq}v_{th}\right)\\
          0 &\:\left(\bm{v}^l\left(t\right){<}v_{th}\right)
        \end{alignedat}
      \right. 
    \end{split} \label{eq:outputSpike}
  \end{equation}
  \begin{equation}
    \begin{split}
      \bm{v}^l\left(t\right)&=h\left(\bm{v}^l\left(t\right)\right)\\
    where\\
    h\left(x\right)&=\left\{
      \begin{alignedat}{2}
        &v_{rset} &\:\left(x{\geq}v_{th}\right)\\
        &x &\:\left(x{<}v_{th}\right)
      \end{alignedat}
    \right. 
    \end{split} \label{eq:outputSpike2}
  \end{equation}

\begin{figure*}[t]
    \centering
    \parbox{.8\linewidth}{
        \centering
        \includesvg[width=1 \linewidth, inkscapelatex=false]{static/sec2_laplace}
        \caption{入力のタイムスケーリングとSNN内部状態ダイナミクス}
        \label{sec2:fig:inspike}
    }

\end{figure*}
\makeatletter % @が使える
\subsection{入力スパイク列のタイムスケール変動とSNNの内部状態ダイナミクスの関係}

入力スパイク列$o(t)$のタイムスケール変動に対して, 時定数$\tau$と膜抵抗$r$を変化させることで, SNNの内部状態$v(t)$のダイナミクスの変化を抑えられることを示す (\Fig{sec2:fig:inspike}).
まず, 入力スパイク列のタイムスケールが変動したときの理想的なSNN内部状態$v$についてである.
理想的には, 入力の時間方向への$a$倍のスケーリングに対して, SNN内部状態のダイナミクスは変化せず, 単純な時間方向の伸縮のみが生じると考えられる.
LIFモデルを用いて理想状態を表すと\refeqn{sec2:eq:ideal}のようになる.

\begin{equation}
    \begin{split}
        % \tau \frac{dv(at)}{dt} &= -(v(at)-v_{rest}) + ri(at)\\
        \tau \frac{dv(at)}{dt} &= -(v(at)-v_{rest}) + r(w o(at) + b)
    \end{split}
    \label{sec2:eq:ideal}
\end{equation}

ここで, $o(at),~v(at)$はそれぞれ時間方向に$a$倍された入力スパイクとSNN内部状態である.
また, $\tau,~r$は理想状態における時定数と膜電位であり, 簡単のためSNNの1つのニューロンについてのみ考える.
ラプラス変換によって\refeqn{sec2:eq:ideal}を$s$領域に変換すると\refeqn{sec2:eq:ideal_laplace}のように表せる.

\begin{equation}
    \begin{split}
        V(\frac{s}{a}) &= \frac{1}{\tau s+1}(\frac{a v_{rest}}{s}+r w O(\frac{s}{a})+\frac{ar}{s}b)\\
        &= \frac{1}{\tau s+1}(\frac{a v_{rest}}{s}+r w +\frac{ar}{s}b)
    \end{split}
    \label{sec2:eq:ideal_laplace}
\end{equation}
ここで, $v(t),~o(t)$のラプラス変換を$V(s), ~ O(s)$とする.
さらに, 入力スパイク列$o(t)$は0か1の離散パルス値であり, ディラックのデルタ関数$\delta(t)$で表現できる\cite{Henkes2024}ため$O(s)\simeq1$と近似している.

次に, 実際のSNNの内部状態についてである.
実際の内部状態は, 入力に対して単純なタイムスケーリングが生じるとは限らない.
そのため, 内部状態は$v(t)$と表され, LIFモデルを用いると\refeqn{sec2:eq:actual}のように表せる.
\begin{equation}
    \begin{split}
        % \tau \frac{dv(at)}{dt} &= -(v(at)-v_{rest}) + ri(at)\\
        \tau' \frac{dv(t)}{dt} &= -(v(t)-v_{rest}) + r'(w o(at) + b)
    \end{split}
    \label{sec2:eq:actual}
\end{equation}
ここで,$\tau',~r'$は実際の状態における時定数と膜抵抗である.
理想的な場合と同様に, \refeqn{sec2:eq:actual}に対してラプラス変換を行うと\refeqn{sec2:eq:actual_laplace}となる.
\begin{equation}
    \begin{split}
        V(s) &= \frac{1}{\tau' s+1}(\frac{v_{rest}}{s}+\frac{r'}{a} w O(\frac{s}{a})+\frac{r'}{s}b)
    \end{split}
    \label{sec2:eq:actual_laplace}
\end{equation}
さらに, \refeqn{sec2:eq:actual_laplace}における$s$を$s/a$に置き換え, $O(s)\simeq1$の近似を用いると, \refeqn{sec2:eq:actual_laplace2}となる.
\begin{equation}
    \begin{split}
        V(\frac{s}{a}) &= \frac{1}{\frac{\tau'}{a} s+1}(\frac{av_{rest}}{s}+\frac{r'}{a} w+\frac{a r'}{s}b)
    \end{split}
    \label{sec2:eq:actual_laplace2}
\end{equation}

ここで, 実際の内部状態式\refeqn{sec2:eq:actual_laplace2}と理想の内部状態式\refeqn{sec2:eq:ideal_laplace}を比較することで, SNNの内部状態を理想状態に変換するための条件が得られる (\refeqn{sec2:eq:condition}).
\begin{equation}
    \begin{split}
        \tau'&=a ~\tau\\
        r'&=a~r \\
        b&=0
    \end{split}
    \label{sec2:eq:condition}
\end{equation}
まずは, SNNを用いた推論モデルをバイアス$\bm{b}=0$のニューラルネットワークで構成する.
また推論時には, $a$倍の入力のタイムスケーリングに対して, SNNの時定数および膜抵抗を$a$倍に変動させる.
このような条件をSNNに与えることで, 内部状態ダイナミクスの変動を抑制し, 単純なタイムスケーリングを生じさせることが可能であると言える.
結果として, 推論モデルの出力が入力の時間変動によらなくなり, 多様なタイムスケールへの頑健性が向上すると考えられる.

\makeatletter % @が使える
\subsection{時定数と膜抵抗の推定およびSNNへの適用}
入力スパイク列$\bm{o(t)}$に対して, \refeqn{sec2:eq:condition}の条件を満たす時定数$\tau',~r'$を線形回帰モデルを用いて推定する.
その後, SNNの時定数と膜抵抗を推定した$\tau',~r'$に適用することで, 入力の時間変動に対して頑健な推論を行う.

まず, 入力スパイク列$o(t)$のタイムスケール$a$をスパイク密度$\rho$を用いて推定する.
スパイク密度$\rho$は, ある一定時間$T$のうちに$o(t)=1$である割合を表す.
推定は線形回帰を用いて近似を行う (\refeqn{sec2:eq:reg}).
\begin{equation}
    \begin{split}
        \rho&=\frac{1}{T} \frac{1}{\Pi_{k=1}^{N}d_k} \sum_t^T \sum_{i_i,..,i_N}o_{t,i_1,..,i_N}\\
        \log{\hat{a}}&=\alpha \log{\rho} + \beta
    \end{split}
    \label{sec2:eq:reg}
\end{equation}
ここで, スパイク列$\bm{o}$は$T \times N$次元のテンソルとし, $d_i$は各次元のサイズである.
また, $\alpha, \beta$は線形回帰によって求めた値であり, $\hat{a}$は推定したタイムスケールである.

次に, 推定した$\hat{a}$を用いて, SNNの時定数と膜抵抗を\refeqn{sec2:eq:condition}を満たす値に変動させる.
基準となるSNNの時定数と膜抵抗を$\tau_{base},~r_{base}$とすると, \refeqn{sec2:eq:replace}のように表される.
\begin{equation}
    \begin{split}
        \tau_{base} \rightarrow \hat{a}~\tau_{base}\\
        r_{base} \rightarrow \hat{a}~r_{base}\\
    \end{split}
    \label{sec2:eq:replace}
\end{equation}