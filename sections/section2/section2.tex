\section{手法}

本手法で用いるモデル構造を\Fig{sec2:fig:model}に示す.
本モデルは大きく2つの構成に分かれる.
\begin{itemize}
    \item SNNの内部状態ダイナミクスを決定する時定数・膜抵抗の推定器
    \item SNN推論器
\end{itemize}
SNNの内部状態ダイナミクスを決定する時定数や膜抵抗は, 推論時には固定した値を用いることが多い\cite{ParametricSNN}\cite{zheng2024temporal}.
本研究ではこれらのパラメータの逐次推定器を用いることで, SNNが基準となるタイムスケールの学習のみによって, 多様なタイムスケールでの推論が可能になることを示す.
% 図をきれいに配置したいだけならparboxを使う (Fgi1, Fig2としたいとき)
% 関連する図をきれいに配置するときはminipageを使う (Fig1(a), (b)としたいとき)

\begin{figure*}[htbp]
    \centering

    \parbox{.9 \linewidth}{
        \centering
        \includegraphics[width=1 \linewidth]{static/samples/dummy.png}
        \caption{モデル構造}
        \label{sec2:fig:model}
    }
    
    \vspace{1em} % 図の間にスペースを追加

    \parbox{.45\linewidth}{
        \centering
        \includegraphics[width=1 \linewidth]{static/samples/dummy.png}
        \caption{SNNの流れ}
        \label{sec2:fig:snn}
    }
    \parbox{.45\linewidth}{
        \centering
        \includegraphics[width=1 \linewidth]{static/samples/dummy.png}
        \caption{入力スパイクのタイムスケーリング}
        \label{sec2:fig:inspike}
    }
\end{figure*}
\makeatletter % @が使える
\subsection{Spiking Neural Network}
Spiking Neural Network (SNN) は脳の神経細胞のダイナミクスを模倣し, 0か1のスパイク値を入出力とするニューラルネットワークである.
時刻$t$における第$l-1$層から$l$層のSNNへの入力$\bm{o^{l-1}(t)}$は, \Fig{sec2:fig:snn}に示すように処理され, 次の第$l+1$層へ出力される.

% \begin{enumerate}
%     \item 入力スパイク$\bm{o^{l-1}(t)}$への重み付け
%     \item 入力に基づくSNNの内部状態の更新
%     \item 内部状態に基づくスパイク$\bm{o^l(t)}$の出力
% \end{enumerate}


まず, SNNへ入力されたスパイク$\bm{o^{l-1}(t)}$は\refeqn{eq:input_spike}によって重み付けされシナプス電流$\bm{i^l(t)}$へ変換される.

\begin{equation}
    \bm{i^l(t)} = \bm{W^l}\bm{o^{l-1}(t)} + \bm{b^l}
    \label{eq:input_spike}
\end{equation}
ここで, $\bm{W^l}, \bm{b^l}$はそれぞれ第$l$層のニューラルネットワークの重みとバイアスである.

次に, 第$l$層のSNNの内部状態$\bm{v^l(t)}$は, 神経細胞の活動をモデル化したLeaky Integrate-and-Fire (LIF) モデル (\refeqn{eq:lif}) によって更新される.

\begin{equation}
    {\tau}\frac{d\bm{v^l(t)}}{dt}=-(\bm{v^l({t-1})}-v_{rest})+r\bm{i^l(t)}
    \label{eq:lif}
\end{equation}
ここで, $\tau$は神経細胞の時定数, $v_{rest}$は内部状態の初期状態, $r$は神経細胞の膜抵抗である.

最後に, 内部状態$v^l(t)$が一定の閾値$v_{th}$を超えたときに出力スパイク$\bm{o^l(t)}$が1となって出力される (\refeqn{eq:outputSpike}).
また, 閾値を超えた内部状態は初期状態へとリセットされる (\refeqn{eq:outputSpike2}).
\begin{equation}
    \begin{split}
      \bm{o(t)^{l}}&=\left\{
        \begin{alignedat}{2}
          1 &\:(\bm{v^l(t)}{\geq}v_{th})\\
          0 &\:(\bm{v^l(t)}{<}v_{th})
        \end{alignedat}
      \right. 
    \end{split} \label{eq:outputSpike}
  \end{equation}
  \begin{equation}
    \begin{split}
      \bm{v^l(t)}&=h(\bm{v^l(t)})\\
    where\\
    h(x)&=\left\{
      \begin{alignedat}{2}
        &v_{rset} &\:(x{\geq}v_{th})\\
        &x &\:(x{<}v_{th})
      \end{alignedat}
    \right. 
    \end{split} \label{eq:outputSpike2}
  \end{equation}

\makeatletter % @が使える
\subsection{SUBTITLE}
ああああああああああああああああああああああああああああああああああああああああああああああああああ
\makeatletter % @が使える
\subsection{逐次的な時定数および膜抵抗の推定とSNNへの適用}
入力スパイク列$\bm{o(t)}$に対して\refeqn{sec2:eq:condition}の条件を満たす時定数$\tau',~r'$を逐次的に推定する.
SNNの時定数と膜抵抗を推定した$\tau',~r'$に置き換えることで, 入力のタイムスケール変化に対して頑健な推論を行う.

まず, 入力スパイク列$o(t)$のタイムスケール$a$をスパイク密度$\rho$を用いて推定する.
スパイク密度$\rho$は, ある一定時間$T$のうちに$o(t)=1$であった割合を表す.
図XXXに示すようにタイムスケールの伸縮に対してスパイク密度は変動する.
そのため, スパイク密度$\rho$を用いてタイムスケール$a$の推定が可能だと考えられる.
本研究では線形回帰を用いて近似を行う (\refeqn{sec2:eq:reg}).
\begin{equation}
    \begin{split}
        \rho&=\frac{1}{T} \frac{1}{\Pi_{k=1}^{N}d_k} \sum_t^T \sum_{i_i,..,i_N}o_{t,i_1,..,i_N}\\
        \log{\hat{a}}&=\alpha \log{\rho} + \beta
    \end{split}
    \label{sec2:eq:reg}
\end{equation}
ここで, スパイク列$\bm{o}$は$T \times N$次元のテンソルとし, $d_i$は各次元のサイズである.
また, $\alpha, \beta$は線形回帰によって求めた値であり, $\hat{a}$は推定したタイムスケールである.

次に, 推定した$\hat{a}$を用いて, SNNの時定数と膜抵抗を\refeqn{sec2:eq:condition}を満たす値に変動させる.
基準となるSNNの時定数と膜抵抗を$\tau_{base},~r_{base}$とすると, \refeqn{sec2:eq:replace}のように表される.
\begin{equation}
    \begin{split}
        \tau_{base} \rightarrow \hat{a}~\tau_{base}\\
        r_{base} \rightarrow \hat{a}~r_{base}\\
    \end{split}
    \label{sec2:eq:replace}
\end{equation}