\makeatletter % @が使える
\subsection{入力の時間変動と内部状態の関係}

入力スパイク列$o\left(t\right)$のタイムスケール変動に対して, 時定数$\tau$と膜抵抗$r$を変化させることで, SNNの内部状態$v\left(t\right)$のダイナミクスの変化を抑えられることを示す (\Fig{sec2:fig:inspike}).
まず, 入力スパイク列のタイムスケールが変動したときの理想的なSNN内部状態$v$についてである.
理想的には, 入力の時間方向への$a$倍のスケーリングに対して, SNN内部状態のダイナミクスは変化せず, 単純な時間方向の伸縮のみが生じると考えられる.
LIFモデルを用いて理想状態を表すと\refeqn{sec2:eq:ideal}のようになる.

\begin{equation}
    \begin{split}
        % \tau \frac{dv\left(at)}{dt} &= -\left(v\left(at)-v_{rest}) + ri\left(at)\\
        \tau \frac{dv\left(at\right)}{dt} &= -\left(v\left(at\right)-v_{rest}\right) + r\left(w o\left(at\right) + b\right)
    \end{split}
    \label{sec2:eq:ideal}
\end{equation}

ここで, $o\left(at\right),~v\left(at\right)$はそれぞれ時間方向に$a$倍された入力スパイクと内部状態である.
また, $\tau,~r$は理想状態における時定数と膜電位であり, 簡単のためSNNの1つのニューロンについてのみ考える.
ラプラス変換により\refeqn{sec2:eq:ideal}を$s$領域に変換すると\refeqn{sec2:eq:ideal_laplace}となる.

\begin{equation}
    \begin{split}
        V\left(\frac{s}{a}\right) &= \frac{1}{\tau s+1}\left(\frac{a v_{rest}}{s}+r w O\left(\frac{s}{a}\right)+\frac{ar}{s}b\right)\\
        &= \frac{1}{\tau s+1}\left(\frac{a v_{rest}}{s}+r w +\frac{ar}{s}b\right)
    \end{split}
    \label{sec2:eq:ideal_laplace}
\end{equation}
ここで, $v\left(t\right),~o\left(t\right)$のラプラス変換を$V\left(s\right), ~ O\left(s\right)$とする.
さらに, 入力スパイク列$o\left(t\right)$は0か1の離散パルス値であり, ディラックのデルタ関数$\delta\left(t\right)$で表現できる\cite{Henkes2024}ため$O\left(s\right)\simeq1$と近似している.

次に, 実際のSNNの内部状態についてである.
実際の内部状態は, 入力に対して単純なタイムスケーリングが生じるとは限らない.
そのため, 内部状態は$v\left(t\right)$と表され, LIFモデルを用いると\refeqn{sec2:eq:actual}のように表せる.
\begin{equation}
    \begin{split}
        % \tau \frac{dv\left(at\right)}{dt} &= -\left(v\left(at\right)-v_{rest}\right) + ri\left(at\right)\\
        \tau' \frac{dv\left(t\right)}{dt} &= -\left(v\left(t\right)-v_{rest}\right) + r'\left(w o\left(at\right) + b\right)
    \end{split}
    \label{sec2:eq:actual}
\end{equation}
ここで,$\tau',~r'$は実際の状態における時定数と膜抵抗である.
理想的な場合と同様に, \refeqn{sec2:eq:actual}に対してラプラス変換を行うと\refeqn{sec2:eq:actual_laplace}となる.
\begin{equation}
    \begin{split}
        V\left(s\right) &= \frac{1}{\tau' s+1}\left(\frac{v_{rest}}{s}+\frac{r'}{a} w O\left(\frac{s}{a}\right)+\frac{r'}{s}b\right)
    \end{split}
    \label{sec2:eq:actual_laplace}
\end{equation}
さらに, \refeqn{sec2:eq:actual_laplace}における$s$を$s/a$に置き換え, $O\left(s\right)\simeq1$の近似を用いると, \refeqn{sec2:eq:actual_laplace2}となる.
\begin{equation}
    \begin{split}
        V\left(\frac{s}{a}\right) &= \frac{1}{\frac{\tau'}{a} s+1}\left(\frac{av_{rest}}{s}+\frac{r'}{a} w+\frac{a r'}{s}b\right)
    \end{split}
    \label{sec2:eq:actual_laplace2}
\end{equation}

ここで, 実際の内部状態式\refeqn{sec2:eq:actual_laplace2}と理想の内部状態式\refeqn{sec2:eq:ideal_laplace}を比較することで, SNNの内部状態を理想状態に変換するための条件が得られる (\refeqn{sec2:eq:condition}).
\begin{equation}
    \begin{split}
        \tau'&=a ~\tau\\
        r'&=a~r \\
        b&=0
    \end{split}
    \label{sec2:eq:condition}
\end{equation}
まずは, SNNを用いた推論モデルをバイアス$\bm{b}=0$のニューラルネットワークで構成する.
また推論時には, $a$倍の入力のタイムスケーリングに対して, SNNの時定数および膜抵抗を$a$倍に変動させる.
このような条件をSNNに与えることで, 内部状態ダイナミクスの変動を抑制し, 単純なタイムスケーリングを生じさせることが可能であると言える.
結果として, 推論モデルの出力が入力の時間変動によらなくなり, 多様なタイムスケールへの頑健性が向上すると考えられる.
