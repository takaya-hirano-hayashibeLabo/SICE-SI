\makeatletter % @が使える
\subsection{Spiking Neural Network}
Spiking Neural Network (SNN) は脳の神経細胞のダイナミクスを模倣し, 0か1のスパイク値を入出力とするニューラルネットワークである.
時刻$t$における第$l-1$層から$l$層のSNNへの入力スパイク$\bm{o_t^{l-1}}$は, 以下の3つのステップで処理され, 次の第$l+1$層へ出力される (\Fig{sec2:fig:snn}).

\begin{enumerate}
    \item 入力スパイク$\bm{o_t^{l-1}}$への重み付け
    \item 入力に基づくSNNの内部状態の更新
    \item 内部状態に基づくスパイク$\bm{o_t^l}$の出力
\end{enumerate}


まず, SNNへ入力されたスパイク$\bm{o_t^{l-1}}$は\refeqn{eq:input_spike}によって重み付けされシナプス電流$\bm{i_t^l}$へ変換される.

\begin{equation}
    \bm{i_t^l} = \bm{W^l}\bm{o_t^{l-1}} + \bm{b^l}
    \label{eq:input_spike}
\end{equation}
ここで, $\bm{W^l}, \bm{b^l}$はそれぞれ第$l$層のニューラルネットの重みとバイアスである.

次に, 第$l$層のSNNの内部状態$\bm{v_t^l}$は, 神経細胞の活動をモデル化したLeaky Integrate-and-Fire (LIF) モデル (\refeqn{eq:lif}) によって更新される.

\begin{equation}
    {\tau}\frac{d\bm{v_t^l}}{dt}=-(\bm{v_{t-1}^l}-v_{rest})+r\bm{i_t^l}
    \label{eq:lif}
\end{equation}
ここで, $\tau$は神経細胞の時定数, $v_{rest}$は内部状態の初期状態, $r$は神経細胞の膜抵抗である.

最後に, 内部状態$v_t^l$が一定の閾値$v_{th}$を超えたときに出力スパイク$\bm{o_t^l}$が1となって出力される (\refeqn{eq:outputSpike}).
また, 閾値を超えた内部状態は初期状態へとリセットされる (\refeqn{eq:outputSpike2}).
\begin{equation}
    \begin{split}
      \bm{o_t^{l}}&=\left\{
        \begin{alignedat}{2}
          1 &\:(\bm{v_t^l}{\geq}v_{th})\\
          0 &\:(\bm{v_t^l}{\leq}v_{th})
        \end{alignedat}
      \right. 
    \end{split} \label{eq:outputSpike}
  \end{equation}
  \begin{equation}
    \begin{split}
      \bm{v_t^l}&=h(\bm{v_t^l})\\
    where\\
    h(x)&=\left\{
      \begin{alignedat}{2}
        &v_{rset} &\:(x{\geq}v_{th})\\
        &x &\:(x{\leq}v_{th})
      \end{alignedat}
    \right. 
    \end{split} \label{eq:outputSpike2}
  \end{equation}
