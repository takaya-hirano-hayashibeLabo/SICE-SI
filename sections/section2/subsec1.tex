\makeatletter % @が使える
\subsection{Spiking Neural Network}
Spiking Neural Network (SNN) は脳の神経細胞のダイナミクスを模倣し, 0か1のスパイク値を入出力とするニューラルネットワークである.
時刻$t$における第$l-1$層から$l$層のSNNへの入力$\bm{o}^{l-1}\left(t\right)$は, \Fig{sec2:fig:snn}に示すように処理され, 次の第$l+1$層へ出力される.

% \begin{enumerate}
%     \item 入力スパイク$\bm{o^{l-1}\left(t\right)}$への重み付け
%     \item 入力に基づくSNNの内部状態の更新
%     \item 内部状態に基づくスパイク$\bm{o^l\left(t\right)}$の出力
% \end{enumerate}


まず, SNNへ入力されたスパイク$\bm{o}^{l-1}\left(t\right)$は\refeqn{eq:input_spike}によって重み付けされシナプス電流$\bm{i}^l\left(t\right)$へ変換される.

\begin{equation}
    \bm{i}^l\left(t\right) = \bm{W}^l\bm{o}^{l-1}\left(t\right) + \bm{b}^l
    \label{eq:input_spike}
\end{equation}
ここで, $\bm{W}^l, \bm{b}^l$はそれぞれ第$l$層のニューラルネットワークの重みとバイアスである.

次に, 第$l$層のSNNの内部状態$\bm{v}^l\left(t\right)$は, 神経細胞の活動をモデル化したLeaky Integrate-and-Fire (LIF) モデル (\refeqn{eq:lif}) によって更新される.

\begin{equation}
    {\tau}\frac{d\bm{v}^l\left(t\right)}{dt}=-\left(\bm{v}^l\left({t-1}\right)-v_{rest}\right)+r\bm{i}^l\left(t\right)
    \label{eq:lif}
\end{equation}
ここで, $\tau$は神経細胞の時定数, $v_{rest}$は内部状態の初期状態, $r$は神経細胞の膜抵抗である.

最後に, 内部状態$\bm{v}^l\left(t\right)$が一定の閾値$v_{th}$を超えたときに出力スパイク$\bm{o}^l\left(t\right)$が1となって出力される (\refeqn{eq:outputSpike}).
また, 閾値を超えた内部状態は初期状態へとリセットされる (\refeqn{eq:outputSpike2}).
\begin{equation}
    \begin{split}
      \bm{o}\left(t\right)^{l}&=\left\{
        \begin{alignedat}{2}
          1 &\:\left(\bm{v}^l\left(t\right){\geq}v_{th}\right)\\
          0 &\:\left(\bm{v}^l\left(t\right){<}v_{th}\right)
        \end{alignedat}
      \right. 
    \end{split} \label{eq:outputSpike}
  \end{equation}
  \begin{equation}
    \begin{split}
      \bm{v}^l\left(t\right)&=h\left(\bm{v}^l\left(t\right)\right)\\
    where\\
    h\left(x\right)&=\left\{
      \begin{alignedat}{2}
        &v_{rset} &\:\left(x{\geq}v_{th}\right)\\
        &x &\:\left(x{<}v_{th}\right)
      \end{alignedat}
    \right. 
    \end{split} \label{eq:outputSpike2}
  \end{equation}
