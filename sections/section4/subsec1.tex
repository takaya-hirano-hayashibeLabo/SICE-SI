\makeatletter % @が使える
\subsection{実験1 入力のタイムスケール変動とSNN内部状態変化}

\Fig{sec4:fig:exp1}にある入力のタイムスケール変動とSNNの内部状態変化を示す.
$o(t), ~ v(t)$はそれぞれ, 基準となる入力スパイクとSNN内部状態変化である.
$v(at)$は$v(t)$を線形補間によって$a$倍のタイムスケーリングした理想的な内部状態である.
また, $o(at)$は$o(t)$を$a$倍タイムスケールした入力スパイクである.
$v_{LIF}(t),~ v_{proposed}(t)$はそれぞれ, $o(at)$が入力されたとき通常のSNNの内部状態, 条件\refeqn{sec2:eq:condition}を満たすSNNの内部状態である.

\Table{sec4:tab:exp1}に$v(t)$と$v_{LIF}(t),~ v_{proposed}(t)$それぞれとのMSEを示す.
\begin{table}[htb]
    \centering
    \caption{基準膜電位との平均二乗誤差}
    \label{sec4:tab:exp1}
    \begin{tabular}{cc}
        \hline
        $\mathbf{\mathrm{MSE}(\mathit{v,\ v_{LIF}})}$ & $\mathbf{\mathrm{MSE}(\mathit{v,\ v_{proposed}})}$\\
        \hline
        $(1.41 \pm 5.67) \times 10^{-4}$   & $(2.43\pm 5.26 )\times 10^{-6}$
    \end{tabular}
\end{table}

\textbf{\Table{sec4:tab:exp1}}より, 提案手法は通常のSNNの膜電位変化と比較して約58倍MSEが小さくなることがわかる.
また, \textbf{\Fig{sec4:fig:exp1}}より, $v_{LIF}$は入力のタイムスケールに伴って, そのダイナミクスが変化することがわかる.
一方で, $v_{proposed}$はダイナミクスの変化は小さく, 単純なタイムスケールに近い状態であることがわかる.
これらの結果から, 入力のタイムスケール変動に従って, 条件\refeqn{sec2:eq:condition}を満たすように時定数$\tau$と膜抵抗$r$を変化させることで, SNNの内部状態ダイナミクスを変えずにタイムスケーリングすることが可能であると言える.