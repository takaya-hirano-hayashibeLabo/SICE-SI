\documentclass{sice-si}

\usepackage{bm} %bm使えるように
\usepackage{tabularx} % プリアンブルに追加
\usepackage{array} % プリアンブルに追加

\usepackage{svg}
\usepackage{fontspec}

\AtBeginBibliography{\fontsize{9}{10}\selectfont} %参考文献のフォントサイズを8ptに

\def\LuaLaTeX{Lua\LaTeX}
\def\upLaTeX{up\LaTeX}
\def\pTeX{p\TeX}
\def\SICEcls{sice.cls}

\newcommand{\TeXEngine}{unknown}
\ifluatex
    \renewcommand{\TeXEngine}{\LuaLaTeX}
\else
    \ifuptex
        \renewcommand{\TeXEngine}{\upLaTeX}
    \fi
\fi

\addbibresource{references.bib}

% タイトルと著者名
\title{\textbf{時定数と膜抵抗の動的変動による \\ 広範な時間スケールに頑健なSpiking Neural Networkの構築}} % 和文タイトル
\name{○平野 貴也 (東北大学), 沓澤 京 (東北大学), 大脇 大 (東北大学), 林部 充宏 (東北大学)} % 著者名
\etitle{A Spiking Neural Network Robust to a Wide Range of Time Scales \\ through Dynamic Variations in Time Constants and Membrane Resistance} % 英文タイトル
\ename{○Takaya HIRANO (Tohoku University), Kyo KUTSUZAWA (Tohoku University),\\ Dai OWAKI (Tohoku University), and Mitsuhiro HAYASHIBE (Tohoku University)}	%著者名(英)

\begin{document}
% アブストラクト (60words以内)
\abst{
    Spiking Neural Networks (SNNs) are similar to brain neural circuits, improving memory efficiency and reducing energy use. 
    However, traditional SNNs with fixed time constants struggle with varying temporal characteristics. 
    This study introduces SNNs with dynamic parameters, including time constants, enabling adaptation to multiple time scales from single-scale training, maintaining high accuracy and robustness across different time scales.
    }

\date{} % 日付を出力しない
% タイトルの出力
\maketitle

\thispagestyle{empty}
\pagestyle{empty}



% >> 内容 >>
% \input{sections/sample.tex}
\section{緒言}
Spiking Neural Network (SNN) は, Artificial Neural Network (ANN) と比べて, より脳の神経回路を模してモデル化されたニューラルネットワークである\cite{TAVANAEI201947}.
脳のスパース性を導入することで, SNNはANNに比べて計算時のメモリ効率とエネルギー消費量を大幅に改善することが可能となるため, 近年では第3のニューラルネットワークとして注目されている\cite{Henkes2024}.
SNNのスパース性は, ニューラルネットワークの層間の情報伝達方式によって生み出される.
ANNは連続値を情報として伝達するのに対し, SNNではスパイクと呼ばれる0か1の情報のみを伝達する.
SNNではこのスパイクを時間的に連続に処理することで, スパース性を持つ脳の神経ダイナミクスを模倣する.

SNNはこのようなスパイクの時系列処理による動的特性を持つため, 時系列情報に対して高い処理能力を持つと考えられている\cite{zheng2024temporal}.
複雑な時系列情報は通常, 可変のタイムスケールと多量の周波数の情報を持つが, 脳は異なるタイムスケールの情報に対して頑健な処理が可能である\cite{10.1162/jocn_a_01615}.
例えば, 異なる速度で話す話者であっても, 人は容易にその認識ができることなどが挙げられる.
これは時系列処理において, マルチスケールの時間ダイナミクスに対して頑健な情報処理が重要であることを示唆している.

しかしながら, 既存のSNNのほとんどは, 異なる時間特性を活用できないLeaky Integrated-and-fire (LIF)モデルを用いて神経細胞ダイナミクスのモデル化を行っている\cite{dayan2003theoretical}.
LIFモデルが多く採用される理由は, 数理的に扱いやすく計算コストが低いためである.
一方で, LIFモデルでは過去の情報をどれだけ維持・忘却するかを表す時定数を固定した値として扱う.
また, 時定数はハイパーパラメータであるため, モデルの設計者が扱う問題によって逐一設定する必要がある.
そのため, LIFモデルを用いたSNNは学習可能なタイムスケールが制限され, 豊富な時間的特性を表現することができない問題がある.

この課題に対して, LIFモデルの時定数に異質性をもたせることで対応する先行研究が多くみられる\cite{10.1145/3407197.3407225}\cite{fang2021incorporating}.
時定数の異質性とは, 固定した値を用いるのではなく, SNNのニューロンごとにばらつきのある値を持たせることである.
Fangら\cite{fang2021incorporating}はSNNの重みとバイアスに加え, 時定数も学習可能とする学習アルゴリズムを提案した.
これによって, 時定数が異質性を持ち SNNの表現力が向上することや時定数が固定されたSNNよりも学習の収束が速いことを示した.
Zhengら\cite{zheng2024temporal}は, 学習可能な時定数の数を増やすことで, より複雑な時間表現を持つSNNを提案した.
結果として, 各ニューロンがより詳細かつ不均一な時間的特性を持つようになり, 音声, 視覚, 脳波認識に対して, 高い精度と頑健性を示した.

このような学習可能な時定数によってSNNの時間特性が向上する一方で, 学習データ量については問題があると考えられる.
学習を通じて時定数に異なる時間特性を持たせるためには, 異なる時間スケールのデータを学習時に与えなくてはならない.
このようなデータは時間的特性以外は同じ情報を持つため学習の効率が悪化する.
そこで, 本研究ではSNNの時定数を動的に変動させることで, 基準のタイムスケールの学習のみによって, マルチタイムスケールに対応可能なSNNを提案する.
時系列データの認識タスクにおいて, 提案手法は既存手法と比較して, 学習時と異なるタイムスケールを与えた場合でも高い精度を維持する結果となった.
また, 入力の時系列データのタイムスケールが途中で変動する場合においても, 認識精度の低下を抑制する特性を示した.



\section{手法}
\makeatletter % @が使える
\subsection{Spiking Neural Network}
Spiking Neural Network (SNN) は脳の神経細胞のダイナミクスを模倣し, 0か1のスパイク値を入出力とするニューラルネットワークである.
時刻$t$における第$l-1$層から$l$層のSNNへの入力$\bm{o^{l-1}(t)}$は, \Fig{sec2:fig:snn}に示すように処理され, 次の第$l+1$層へ出力される.

% \begin{enumerate}
%     \item 入力スパイク$\bm{o^{l-1}(t)}$への重み付け
%     \item 入力に基づくSNNの内部状態の更新
%     \item 内部状態に基づくスパイク$\bm{o^l(t)}$の出力
% \end{enumerate}


まず, SNNへ入力されたスパイク$\bm{o^{l-1}(t)}$は\refeqn{eq:input_spike}によって重み付けされシナプス電流$\bm{i^l(t)}$へ変換される.

\begin{equation}
    \bm{i^l(t)} = \bm{W^l}\bm{o^{l-1}(t)} + \bm{b^l}
    \label{eq:input_spike}
\end{equation}
ここで, $\bm{W^l}, \bm{b^l}$はそれぞれ第$l$層のニューラルネットワークの重みとバイアスである.

次に, 第$l$層のSNNの内部状態$\bm{v^l(t)}$は, 神経細胞の活動をモデル化したLeaky Integrate-and-Fire (LIF) モデル (\refeqn{eq:lif}) によって更新される.

\begin{equation}
    {\tau}\frac{d\bm{v^l(t)}}{dt}=-(\bm{v^l({t-1})}-v_{rest})+r\bm{i^l(t)}
    \label{eq:lif}
\end{equation}
ここで, $\tau$は神経細胞の時定数, $v_{rest}$は内部状態の初期状態, $r$は神経細胞の膜抵抗である.

最後に, 内部状態$v^l(t)$が一定の閾値$v_{th}$を超えたときに出力スパイク$\bm{o^l(t)}$が1となって出力される (\refeqn{eq:outputSpike}).
また, 閾値を超えた内部状態は初期状態へとリセットされる (\refeqn{eq:outputSpike2}).
\begin{equation}
    \begin{split}
      \bm{o(t)^{l}}&=\left\{
        \begin{alignedat}{2}
          1 &\:(\bm{v^l(t)}{\geq}v_{th})\\
          0 &\:(\bm{v^l(t)}{<}v_{th})
        \end{alignedat}
      \right. 
    \end{split} \label{eq:outputSpike}
  \end{equation}
  \begin{equation}
    \begin{split}
      \bm{v^l(t)}&=h(\bm{v^l(t)})\\
    where\\
    h(x)&=\left\{
      \begin{alignedat}{2}
        &v_{rset} &\:(x{\geq}v_{th})\\
        &x &\:(x{<}v_{th})
      \end{alignedat}
    \right. 
    \end{split} \label{eq:outputSpike2}
  \end{equation}

\makeatletter % @が使える
\subsection{SUBTITLE}
ああああああああああああああああああああああああああああああああああああああああああああああああああ
\makeatletter % @が使える
\subsection{逐次的な時定数および膜抵抗の推定とSNNへの適用}
入力スパイク列$\bm{o(t)}$に対して\refeqn{sec2:eq:condition}の条件を満たす時定数$\tau',~r'$を逐次的に推定する.
SNNの時定数と膜抵抗を推定した$\tau',~r'$に置き換えることで, 入力のタイムスケール変化に対して頑健な推論を行う.

まず, 入力スパイク列$o(t)$のタイムスケール$a$をスパイク密度$\rho$を用いて推定する.
スパイク密度$\rho$は, ある一定時間$T$のうちに$o(t)=1$であった割合を表す.
図XXXに示すようにタイムスケールの伸縮に対してスパイク密度は変動する.
そのため, スパイク密度$\rho$を用いてタイムスケール$a$の推定が可能だと考えられる.
本研究では線形回帰を用いて近似を行う (\refeqn{sec2:eq:reg}).
\begin{equation}
    \begin{split}
        \rho&=\frac{1}{T} \frac{1}{\Pi_{k=1}^{N}d_k} \sum_t^T \sum_{i_i,..,i_N}o_{t,i_1,..,i_N}\\
        \log{\hat{a}}&=\alpha \log{\rho} + \beta
    \end{split}
    \label{sec2:eq:reg}
\end{equation}
ここで, スパイク列$\bm{o}$は$T \times N$次元のテンソルとし, $d_i$は各次元のサイズである.
また, $\alpha, \beta$は線形回帰によって求めた値であり, $\hat{a}$は推定したタイムスケールである.

次に, 推定した$\hat{a}$を用いて, SNNの時定数と膜抵抗を\refeqn{sec2:eq:condition}を満たす値に変動させる.
基準となるSNNの時定数と膜抵抗を$\tau_{base},~r_{base}$とすると, \refeqn{sec2:eq:replace}のように表される.
\begin{equation}
    \begin{split}
        \tau_{base} \rightarrow \hat{a}~\tau_{base}\\
        r_{base} \rightarrow \hat{a}~r_{base}\\
    \end{split}
    \label{sec2:eq:replace}
\end{equation}
\section{実験方法}

\begin{figure*}[t]
    \centering
    \parbox{.9\linewidth}{
        \centering
        \includesvg[width=1 \linewidth, inkscapelatex=false]{static/sec3_gesture}
        \caption{ジェスチャーデータの例 (右腕を大きく回す動作) : イベント形式のため動作の大きい腕の情報が多く記録されている}
        \label{sec3:fig:gesture}
    }

\end{figure*}

まず, 入力スパイク列の時間的な変動に対して, 時定数と膜抵抗を\refeqn{sec2:eq:ideal_laplace}に従って変化させることで, SNNの内部状態ダイナミクスの変化が抑制されることを示す.
次に, 動画形式のデータセットを用いた分類タスクを行う.
その中で, 提案手法は基準となるタイムスケールの学習のみで, 多様な動画速度に対して頑健な推論が可能であることを示す.
\makeatletter % @が使える
\subsection{実験1 SNNの内部状態ダイナミクスのタイムスケーリング評価}
ランダムな値で重みを初期化したSNNに対し, 基準となるスパイク列$\bm{o_{base}}$と基準から$a$倍のタイムスケーリングしたスパイク列$\bm{o_{scaled}}$を入力する.
$\bm{o_{base}},~\bm{o_{scaled}}$が入力されたときの最終層の内部状態$\bm{v_{base}}, ~ \bm{v_{scaled}}$を比較する.
このとき, \refeqn{sec2:eq:ideal_laplace}の条件をSNNに与えることで$\bm{v_{scaled}}$が$\bm{v_{base}}$に近づくことを実験的に示す.
SNNは入出力を1次元とし, ノード数8の全結合層を2層持つモデルを用いた.
ランダムに生成した100通りのスパイク$o(t)$を入力し, $v(at)$に対する$v_{LIF}(t), ~ v_{proposed}(t)$それぞれの平均二乗誤差 (MSE, Mean Squared Error) を計測した.
SNNのそれぞれのパラメータを\Table{sec3:tab:exp1snn}に示す.


% tabularxによって\linewidthで幅指定可能に (required: tabularx)
% \centering\arraybackslashによって, 各セルの中央に値を表示 (required: array)
\begin{table}[htb]
    \centering
    \caption{SNNモデルパラメータ}
    \label{sec3:tab:exp1snn}
    \begin{tabularx}{0.8\linewidth}{>{\centering\arraybackslash}X>{\centering\arraybackslash}X>{\centering\arraybackslash}X}
        \multicolumn{3}{c}{\textbf{SNN architecture}}\\
        \hline
        \textbf{input size}& \textbf{hidden size} & \textbf{output size}\\
        \hline
        1   & 8, 8 & 1 
    \end{tabularx}

    \begin{tabularx}{0.8\linewidth}{>{\centering\arraybackslash}X>{\centering\arraybackslash}X>{\centering\arraybackslash}X>{\centering\arraybackslash}X>{\centering\arraybackslash}X}
        \multicolumn{5}{c}{\textbf{LIF parameters}}\\
        \hline
        $\bm{dt}$&$\bm{v_{rest}}$&$\bm{v_{th}}$&$\bm{\tau}$&$\bm{r}$\\
        \hline
        0.001&0.0&0.01&0.05&1.0
    \end{tabularx}
\end{table}

\begin{figure*}[t]
    \centering

    \parbox{.85 \linewidth}{
        \centering
        % svgのままコンパイルするならこれ. コンパイルは`lualatex --shell-escape`を使う(required: svg, fontspec)
        \includesvg[width=1 \linewidth, inkscapelatex=false]{static/sec4_exp1}
        \caption{入力のタイムスケール変動とSNNの内部状態変化}
        \label{sec4:fig:exp1}
    }
    
    \vspace{0.5em} % 図の間にスペースを追加

    \parbox{.9\linewidth}{
        \centering

        \begin{minipage}{.48 \linewidth} % Adjusted width to fit side by side
            \centering
            \includesvg[width=1 \linewidth, inkscapelatex=false]{static/sec4_exp2.1}
            \subcaption{シーケンス全体変化に対するモデル精度}
            \label{sec4:fig:exp2:1}
        \end{minipage}
        \hspace{0.02\linewidth} % これがないと横並びにならない
        \begin{minipage}{.48 \linewidth} % Adjusted width to fit side by side
            \centering
            \includesvg[width=1 \linewidth, inkscapelatex=false]{static/sec4_exp2.2}
            \subcaption{シーケンス途中変化に対するモデル精度}
            \label{sec4:fig:exp2:2}    
        \end{minipage}    

        \caption{入力のタイムスケール変動とモデル精度}
    }

\end{figure*} %画像の位置をいい感じにするために, section4の画像をsection3に移動

\makeatletter % @が使える
\subsection{実験2 時間変動に対するモデル性能評価}

\subsubsection{データセット}
モデルの学習データセットとしてDVSGesture\cite{dvsgesture}を用いた.
このデータセットは11種類のジェスチャーを時系列に記録したものである.
また, データ形式はイベント形式で保存されており, 変化が生じた時刻とピクセル位置がペアとして保存されている.
イベントが生じた時刻・ピクセル位置を1, それ以外を0とすることで容易にスパイク列へ変換することが可能であるため, SNNを用いたタスク評価に適したデータセットである\cite{9207109}.


\subsubsection{時系列認識モデルとその学習}
提案モデルに対して, DVSGesuteデータセットを用いた時系列認識タスクを学習させた.
また, 比較対象としてLSTM, SNN, Parametric-SNN\cite{ParametricSNN}で構成したモデルも同様のタスクを学習させた.
Parametric-SNNとは, 時定数を学習可能としたSNNであり, 推論時は学習によって得られた時定数を用いる.
LSTMで構成したモデルは, ResNet\cite{ResNet}を用いてフレームごとの特徴抽出を行い, LSTMによって時系列認識を行った\cite{CNNLSTM}.
SNNを用いたモデルはSNNで残差接続を実現したMS-ResNet\cite{MSResNet}を用いて構成した.
また, 畳み込み層間のドロップアウト率は0.3とした.
それぞれのモデル構成を\Table{tab:sec3:modelarchi}に示す.
LSTM, SNN, Parametric-SNNは, $0.5,~1.0,~2.0$倍の3種類のタイムスケールを用いて学習を行った.
提案モデルは$1.0$倍速のデータのみで学習を行い, 線形回帰モデルは3種類のデータを用いて回帰を行った.
DVSGestureデータセットは, 3 msでサンプリングされ, $1.0$倍速で約1.0秒の長さのものを用いた.
\begin{table}[htb]
    \centering
    \caption{評価モデル構成}
    \label{tab:sec3:modelarchi}

    % \begin{tabularx}{1.1\linewidth}{>{\centering\arraybackslash}X>{\centering\arraybackslash}X>{\centering\arraybackslash}X>{\centering\arraybackslash}X>{\centering\arraybackslash}X}
    \begin{tabular}{ccccc}
        \multicolumn{5}{c}{\textbf{Model architecture using LSTM}}\\
        \hline
        \textbf{Layer} & \textbf{Type} & \textbf{Input} & \textbf{output} & \textbf{Layer Nums} \\
        \hline
        1 & ResNet & 2x32x32 & 12x16x16 & 3 \\
        2 & ResNet & 12x16x16 & 32x8x8 & 3 \\
        3 & Linear & 2048 & 512 & - \\
        4 & LSTM & 512 & 512 & - \\
        5 & LSTM & 512 & 256 & - \\
        6 & Linear & 256 & 11 & - \\
    \end{tabular}
    
    % \begin{tabularx}{1.1\linewidth}{>{\centering\arraybackslash}X>{\centering\arraybackslash}X>{\centering\arraybackslash}X>{\centering\arraybackslash}X>{\centering\arraybackslash}X}
    \begin{tabular}{ccccc}
        \multicolumn{5}{c}{\textbf{Model architecture using SNN}}\\
        \hline
        \textbf{Layer} & \textbf{Type} & \textbf{Input} & \textbf{output} & \textbf{Layer Nums} \\
        \hline
        1 & MS-ResNet & 2x32x32 & 12x16x16 & 3 \\
        2 & MS-ResNet & 12x16x16 & 32x8x8 & 3 \\
        3 & Linear & 2048 & 512 & - \\
        4 & Linear & 512 & 11 & - \\
    \end{tabular}

    \begin{tabularx}{0.8\linewidth}{>{\centering\arraybackslash}X>{\centering\arraybackslash}X>{\centering\arraybackslash}X>{\centering\arraybackslash}X>{\centering\arraybackslash}X}
        \multicolumn{5}{c}{\textbf{LIF parameters}}\\
        \hline
        $\bm{dt}$&$\bm{v_{rest}}$&$\bm{v_{th}}$&$\bm{\tau}$&$\bm{r}$\\
        \hline
        0.003&0.0&0.1&0.006&1.0
    \end{tabularx}

\end{table}


\subsubsection{評価方法}
提案手法の入力の時間的変動に対する頑健性を評価するために以下の2つの実験を行った.
\begin{itemize}
    \item シーケンス全体を$a$倍した際のモデル精度評価
    \item シーケンスの前半, 後半を$a_1,~a_2$倍した際のモデル性能評価
\end{itemize}

まず, 入力シーケンス全体を$a$にタイムスケーリングし, 入力の時間変動に対する各モデルの性能を評価した.
倍率$a$は, $1.0$から$20.0$を$1.0$刻みに変化させた値とした.
次に, 入力シーケンスの前半を$a_1$倍, 後半を$a_2$倍にスケールすることで, シーケンスの途中で時間変動が生じた場合の各モデルの性能を評価した.
$a_1,~a_2$はそれぞれ\Table{sec3:tab:exp2:2}に示す値を用いた.

\begin{table}[htb]
    \centering
    \caption{倍率$a_1,~a_2$の組み合わせ}

    \begin{tabularx}{0.8\linewidth}{>{\centering\arraybackslash}X>{\centering\arraybackslash}X>{\centering\arraybackslash}X}
        \hline
        \textbf{Case}&$\bm{a_1}$&$\bm{a_2}$\\
        \hline
        A&1.0&5.0\\
        B&5.0&1.0\\
        C&10.0&5.0\\
        D&5.0&10.0
    \end{tabularx}

    \label{sec3:tab:exp2:2}

\end{table}
% \makeatletter % @が使える
\subsection{SUBTITLE}
ああああああああああああああああああああああああああああああああああああああああああああああああああ\cite{tmpRef}
\section{結果}
\begin{figure*}[t]
    \centering

    \parbox{.85 \linewidth}{
        \centering
        % svgのままコンパイルするならこれ. コンパイルは`lualatex --shell-escape`を使う(required: svg, fontspec)
        \includesvg[width=1 \linewidth, inkscapelatex=false]{static/sec4_exp1}
        \caption{入力のタイムスケール変動とSNNの内部状態変化}
        \label{sec4:fig:exp1}
    }
    
    \vspace{0.5em} % 図の間にスペースを追加

    \parbox{.9\linewidth}{
        \centering

        \begin{minipage}{.48 \linewidth} % Adjusted width to fit side by side
            \centering
            \includesvg[width=1 \linewidth, inkscapelatex=false]{static/sec4_exp2.1}
            \subcaption{シーケンス全体変化に対するモデル精度}
            \label{sec4:fig:exp2:1}
        \end{minipage}
        \hspace{0.02\linewidth} % これがないと横並びにならない
        \begin{minipage}{.48 \linewidth} % Adjusted width to fit side by side
            \centering
            \includesvg[width=1 \linewidth, inkscapelatex=false]{static/sec4_exp2.2}
            \subcaption{シーケンス途中変化に対するモデル精度}
            \label{sec4:fig:exp2:2}    
        \end{minipage}    

        \caption{入力のタイムスケール変動とモデル精度}
    }

\end{figure*}
\makeatletter % @が使える
\subsection{実験1 : 入力の時間変動とSNN内部状態変化}

入力の時間変動とSNNの内部状態変化を\Fig{sec4:fig:exp1}に示す.
$o(t), ~ v(t)$はそれぞれ, 基準となる入力スパイクとSNN内部状態変化である.
$v(at)$は$v(t)$を線形補間によって$a$倍のタイムスケーリングした内部状態である.
また, $o(at)$は$o(t)$を$a$倍タイムスケールした入力スパイクである.
$v_{LIF}(t),~ v_{proposed}(t)$はそれぞれ, $o(at)$が入力されたとき通常のSNNの内部状態, 提案手法のSNNの内部状態である.

\Table{sec4:tab:exp1}に$v(at)$と$v_{LIF}(t),~ v_{proposed}(t)$それぞれとのMSEを示す.
\begin{table}[htb]
    \centering
    \caption{基準内部状態との平均二乗誤差}
    \label{sec4:tab:exp1}
    \begin{tabular}{cc}
        \hline
        $\mathbf{\mathrm{MSE}(\mathit{v,\ v_{LIF}})}$ & $\mathbf{\mathrm{MSE}(\mathit{v,\ v_{proposed}})}$\\
        \hline
        $(1.41 \pm 5.67) \times 10^{-4}$   & $(2.43\pm 5.26 )\times 10^{-6}$
    \end{tabular}
\end{table}

\textbf{\Table{sec4:tab:exp1}}より, 提案手法は通常のSNNの内部状態変化と比較して約58倍MSEが小さくなる結果となった.
また, \textbf{\Fig{sec4:fig:exp1}}より, $v_{LIF}$は入力の時間変化に伴って, そのダイナミクスが変動することがわかる.
一方で, $v_{proposed}$はダイナミクスの変化が小さく, 単純なタイムスケールに近い状態であることがわかる.
これらの結果から, 入力のタイムスケール変動に従って, 条件\refeqn{sec2:eq:condition}を満たすように時定数$\tau$と膜抵抗$r$を変化させることで, SNNの内部状態ダイナミクスを抑制し, タイムスケーリングすることが可能であると言える.
\makeatletter % @が使える
\subsection{実験2 : 入力の時間変動に対するモデル性能評価}

各モデルの学習後のテスト精度と1epochあたりの学習時間を\Table{sec4:tab:exp2}に示す.
提案手法のモデルは他のモデルと比較して, 短い学習時間であった.
これは提案手法は基準のタイムスケールのみを学習するためである.
提案手法は入力に対して, 時定数と膜電位を推定する回帰モデルを生成する必要があるが, その計算時間は数秒程度であり, ニューラルネットワークの学習時間に対して非常に短いため無視できる.
提案手法は入力の特徴抽出と推論をSNNで行い, 入力の時間変動に対する調整を計算コストの軽い回帰モデルで行うことで, 学習効率の良い特性を示した.
\begin{table}[htb]
    \centering
    \caption{各モデルの学習結果と学習時間}
    \label{sec4:tab:exp2}
    \begin{tabular}{cccc}
        \hline
        \textbf{model}&\textbf{test accuracy}&\textbf{seconds/epoch}\\
        \hline
        LSTM&$0.92\pm0.07$&$120.1\pm5.8$\\
        SNN&$0.86\pm0.06$&$209.4\pm7.2$\\
        ParametricSNN&$0.88\pm0.08$&$224.4\pm7.8$\\
        Proposed method&$0.92\pm0.09$&$94.8\pm3.0$\\
    \end{tabular}
\end{table}

入力の時間変動と各モデルのモデル精度変化を\Fig{sec4:fig:exp2:1}に示す.
LSTMとSNNは, 入力のタイムスケールの増加に従って, モデル精度が大きく低下していることがわかる.
ParametricSNNは, $a=6.0$まではモデル精度の劣化を抑えられているが, それより大きいタイムスケールでは精度が低下した.
一方, 提案手法は$a=10.0$までは精度を維持し, それより大きい場合であっても0.85以上のモデル精度を保つ結果となった.

入力シーケンス内の時間変動とモデル精度の関係を\Fig{sec4:fig:exp2:2}に示す.
学習に使用した$a=1.0$の入力が含まれるcaseA, caseBにおいては, caseBのSNNを除く全てのモデルが0.8以上のモデル精度であった.
一方で, 学習を行っていない入力で構成されるcaseC, caseDにおいては, 提案手法のみが0.8以上のモデル精度を維持する結果となった.
これらの結果から, 提案手法は学習コストを抑えつつ, 入力の時間変動に対して頑健な特性を持つ結果となった.
\makeatletter % @が使える
\subsection{SUBTITLE}
ああああああああああああああああああああああああああああああああああああああああああああああああああ\cite{tmpRef}
\section{考察}

\begin{figure*}
    \centering
    \parbox{1.0\linewidth}{
        \centering

        \begin{minipage}{.48 \linewidth} % Adjusted width to fit side by side
            \centering
            \includesvg[width=1 \linewidth, inkscapelatex=false]{static/sec5_fr_snn}
            \subcaption{SNNの出力層におけるスパイク密度}
            \label{sec5:fig:snn}
        \end{minipage}
        \hspace{0.02\linewidth} % これがないと横並びにならない
        \begin{minipage}{.48 \linewidth} % Adjusted width to fit side by side
            \centering
            \includesvg[width=1 \linewidth, inkscapelatex=false]{static/sec5_fr_dyna}
            \subcaption{提案手法モデルの出力層におけるスパイク密度}
            \label{sec5:fig:dyna}    
        \end{minipage}    

        \caption{出力スパイク密度の比較}
        \subcaption*{上段 : タイムスケール$a=1.0$, 下段 : タイムスケール$a=10.0$}
        \label{sec5:fig:disc}
    }
\end{figure*}

実験2より提案手法を用いることで入力の時間変動が生じた場合でも, モデルの精度低下を抑えることが可能であった.
このような結果になった原因として, 時定数と膜抵抗を動的に変化させることで, 入力の時間変動によるモデルの出力の変化を抑えられたからだと考えられる.

実験2において, あるテストデータを入力した際の通常のSNNと提案手法の出力の比較を\Fig{sec5:fig:disc}に示す.
横軸は正規化された時間, 縦軸は単位時間あたりの出力スパイク密度である.
また, グラフのそれぞれの番号は出力層のニューロン番号に対応する.
スパイク密度はその値の変化が少ないほど, 入力のタイムスケール変動によるモデル出力への影響が小さいことを表す.
\Fig{sec5:fig:disc}より, 提案手法は通常のSNNと比較して, 出力スパイク密度の変化が小さいことがわかる.
特に, \#4, \#10の活動の大きいニューロンにおけるスパイク密度変化が小さい.
これは, 入力の時間変動に応じて, \refeqn{sec2:eq:condition}の条件をモデルに対して動的に与えることで, 出力の変動を抑制したからだと考えられる.
一方で, 提案手法における\#11, \#6のニューロンは, タイムスケールの変化によって, その活性・不活性が変化している.
このような変化が生じた原因としては, 理想的な入力と実データのタイムスケーリングでは, そのデータの形式が異なるからだと考えられる.
実験1で示した提案手法におけるSNNの内部状態ダイナミクスの特性は, 時刻$t$におけるスパイクが$a$倍のタイムスケーリングによって, 時刻$at$に移動するという理想的な状況で現れたものである.
実験2では実データを用いたため, 理想的なタイムスケールとは異なっていたと考えられる.
結果として, \refeqn{sec2:eq:condition}の条件を与えたとしても, 出力の変動を抑制できないニューロンが生じたと考えられる.
そのため, 理想的な入力と実際の入力の差異を小さくするように入力データ形式を変換するエンコーダを作成することで, より入力の時間変動に頑健なモデルを構築できると考えられる.

% \makeatletter % @が使える
\subsection{SUBTITLE}
ああああああああああああああああああああああああああああああああああああああああああああああああああ
% \makeatletter % @が使える
\subsection{SUBTITLE}
ああああああああああああああああああああああああああああああああああああああああああああああああああ
% \makeatletter % @が使える
\subsection{SUBTITLE}
ああああああああああああああああああああああああああああああああああああああああああああああああああ\cite{tmpRef}
\section{結言}

本研究では, 入力の時間変動に対してSNNの時定数と膜抵抗を動的に変化させるモデルを提案した.
結果として, 提案手法は入力の時間変動が生じた場合でも, SNNの内部状態のダイナミクスの変化を抑制できる特性を示した.
また, 実際の時系列データの分類タスクに適用することで, 入力の時間変動に対するモデル精度の低下を抑えられる結果となった.
今後の展望としては, 入力データを理想的な形式に近づけるエンコーダの導入が挙げられる.
エンコーダを用いることで, よりSNNの内部状態変化の抑制が可能になり, 入力の時間変動に対して頑健なモデルの構築が可能になると考えられる.
% \makeatletter % @が使える
\subsection{SUBTITLE}
ああああああああああああああああああああああああああああああああああああああああああああああああああ
% \makeatletter % @が使える
\subsection{SUBTITLE}
ああああああああああああああああああああああああああああああああああああああああああああああああああ
% \makeatletter % @が使える
\subsection{SUBTITLE}
ああああああああああああああああああああああああああああああああああああああああああああああああああ\cite{tmpRef}
% >> 内容 >>

% \bibliographystyle{junsrt} %junsrt → 引用順
\printbibliography[title=参考文献]

\end{document}